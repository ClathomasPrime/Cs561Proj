\documentclass[12pt]{article}


\usepackage{wrapfig}
\usepackage{float}
\usepackage{graphicx}
\usepackage[margin=1in]{geometry} 
\usepackage{amsmath,amsthm,amssymb}
\usepackage[vlined,linesnumbered,ruled,resetcount]{algorithm2e}

\DeclareMathOperator*{\argmax}{arg\,max}
\DeclareMathOperator*{\val}{val}

\renewcommand\thesubsection{(\alph{subsection})}

% \usepackage[noend]{algpseudocode}
% \usepackage{algorithm}
% \usepackage{algorithmicx}

\newcommand{\N}{\mathbb{N}}
\newcommand{\Z}{\mathbb{Z}}

\newtheorem{definition}{Definition}
\newtheorem{lemma}{Lemma}
\newtheorem{theorem}{Theorem}
\newtheorem{conjecture}{Conjecture}

\begin{document}

% \renewcommand{\qedsymbol}{\filledbox}

\title{
  You can't handle the lie: \\
  Catching lying actors in the BGP protocol 
}
\author{
  Clay Thomas\\ claytont@cs.princeton.edu
  \and 
  Gavriel Hirsch\\ gbhirsch@cs.princeton.edu 
}
\maketitle

In many BGP networks, nodes can get their preferred path
by advertising a certain path, but forwarding traffic along another.
However, in many of the example networks from the literature,
other nodes can collectively detect these lies
by comparing the route that is advertised (known by the node
the manipulator lies to)
to the route actually taken
(known by the node the manipulator actually forwards traffic to).
We want to study the conditions under which nodes
can lie without being detected,
as well as the economic incentives of other nodes to collaboratively
check each other (such as providers helping customers detect liars).

Our thinking is primarily based on the routing games model of
\cite{RoutingGames}, where node's objectives are simply to get
better paths to a destination.
Additional consideration is given to the model
of \cite{Attraction}, where nodes seek to attract traffic as
well as get better paths.

\section{Definitions}
  We follow the model of \cite{RoutingGames} and \cite{AgtBookDistributed},
  and start with the following model.

\section{In the \cite{RoutingGames} model, you can always catch a liar}

  Intuitively, if a node lies to get a better path, the node it lies
  to and the node it routs through must be connected.
  The following conjecture, which we hope to prove formally,
  makes this precise.
  \begin{conjecture}
    Suppose No Dispute Wheel holds, but route verification does not, and
    assume that the network is connected.
    Suppose that (assuming other nodes play truthfully)
    a node $m$ can achieve a better path to $d$ by announcing
    a route that does not exist to a node $v$.
    Let $m$'s next hop in the manipulated routing tree be denoted $r$.
    Then there exists a path in the network, not containing $m$,
    between $v$ and $r$.
    Moreover, no node along this path benefits from the manipulation
    performed by $m$.
  \end{conjecture}
  \begin{proof}
    For the sake of contradiction, assume that all paths between $r$
    and $v$ include $m$.
    One of $r$ or $v$ is on the ``same side of $m$'' as the destination
    node $d$.
    There must be a path from $r$ to $d$ not containing $m$,
    because $m$ cannot appear twice on it route to $d$ in the manipulated tree.
    Thus, every path from $v$ to $d$ must contain $m$.

    Let $T$ denote the original routing tree, and for each node $n$
    let $T_n$ denote the path $n$ receives to the destination.
    Let $M_n$ denote the advertised route that $n$ selects in the manipulated
    routing tree, i.e. the route $n$ believes it receives,
    while $\widetilde M$ is the actual manipulated routing tree and 
    $\widetilde M_n$ is the actual route $n$'s traffic follows.
    Note that the path $M_n$ need not actually exist in the graph.
  \end{proof}

  This means that, if all nodes other than $m$ are fully collaborative and
  honest, the nodes will be able to detect $m$'s lie
  by communicating along the links that already exist in the network.


\bibliography{proj}{}
\bibliographystyle{alpha}

\end{document}

\documentclass[12pt]{article}


\usepackage{wrapfig}
\usepackage{float}
\usepackage{graphicx}
\usepackage[margin=1in]{geometry} 
\usepackage{amsmath,amsthm,amssymb}
\usepackage[vlined,linesnumbered,ruled,resetcount]{algorithm2e}

\DeclareMathOperator*{\argmax}{arg\,max}
\DeclareMathOperator*{\val}{val}

\renewcommand\thesubsection{(\alph{subsection})}

% \usepackage[noend]{algpseudocode}
% \usepackage{algorithm}
% \usepackage{algorithmicx}

\newcommand{\N}{\mathbb{N}}
\newcommand{\Z}{\mathbb{Z}}

\newtheorem{definition}{Definition}
\newtheorem{lemma}{Lemma}
\newtheorem{theorem}{Theorem}
\newtheorem{conjecture}{Conjecture}

\begin{document}

% \renewcommand{\qedsymbol}{\filledbox}

\title{
  You Can't Handle the Lie: \\
  Next-Hop Verification in BGP
}
\author{
  Clay Thomas\\ claytont@cs.princeton.edu
  \and 
  Gavriel Hirsch\\ gbhirsch@cs.princeton.edu 
}
\maketitle

\begin{abstract}
  The main goal of this work is to provide evidence for the following
  claim: corroborating facts from the data plane and control plane
  can improve the incentive properties of BGP.
\end{abstract}

  % Needs some rewriting and redistribution::
  In \cite{RoutingGames}, the authors take a popular
  and well-studied extension of BGP,
  and study how it interacts with incentive properties.
  Their central claim of the authors is
  that BGP with path verification (such as S-BGP) has
  simple incentive properties: assuming all other nodes tell the truth,
  no node (or group of nodes) can lie in order to get strictly better routes.
  However, in \cite{Attraction} it is demonstrated that in realistic models
  nodes have incentives to lie in order to \emph{attract traffic}, e.g.
  ISP attracting traffic from customers.
  One idea of \cite{Attraction} is to find the simplest set of criterion
  needed for incentive-compatibility to hold, including the
  simplest form of verification required.
  They find that in some settings, a simple form of verification known
  as \emph{loop verification} suffices.
  Our work is motivated by the question:
  \emph{What is the simplest form of verification needed to provide
  good incentive properties of BGP?}
  We call the form of verification we came up with Next-Hop verification.


\section{Introduction}
\section{Previous Work}
\section{Next-Hop Verification}
  \subsection{Informal Model and Assumptions}
  \subsection{The Protocol}
\section{Our Implementation Sketch}
\section{Conclusion}

\bibliography{proj}{}
\bibliographystyle{alpha}

\appendix

\section{Formal Definitions}
  We follow the model of \cite{RoutingGames} and \cite{AgtBookDistributed},
  and start with the following model.

\section{In the \cite{RoutingGames} model, you can always catch a liar}

  Intuitively, if a node lies to get a better path, the node it lies
  to and the node it routs through must be connected.
  The following conjecture, which we hope to prove formally,
  makes this precise.
  \begin{conjecture}
    Suppose No Dispute Wheel holds, but route verification does not, and
    assume that the network is connected.
    Suppose that (assuming other nodes play truthfully)
    a node $m$ can achieve a better path to $d$ by announcing
    a route that does not exist to a node $v$.
    Let $m$'s next hop in the manipulated routing tree be denoted $r$.
    Then there exists a path in the network, not containing $m$,
    between $v$ and $r$.
    Moreover, no node along this path benefits from the manipulation
    performed by $m$.
  \end{conjecture}
  \begin{proof}
    For the sake of contradiction, assume that all paths between $r$
    and $v$ include $m$.
    One of $r$ or $v$ is on the ``same side of $m$'' as the destination
    node $d$.
    There must be a path from $r$ to $d$ not containing $m$,
    because $m$ cannot appear twice on it route to $d$ in the manipulated tree.
    Thus, every path from $v$ to $d$ must contain $m$.

    Let $T$ denote the original routing tree, and for each node $n$
    let $T_n$ denote the path $n$ receives to the destination.
    Let $M_n$ denote the advertised route that $n$ selects in the manipulated
    routing tree, i.e. the route $n$ believes it receives,
    while $\widetilde M$ is the actual manipulated routing tree and 
    $\widetilde M_n$ is the actual route $n$'s traffic follows.
    Note that the path $M_n$ need not actually exist in the graph.
  \end{proof}

  This means that, if all nodes other than $m$ are fully collaborative and
  honest, the nodes will be able to detect $m$'s lie
  by communicating along the links that already exist in the network.

\end{document}
